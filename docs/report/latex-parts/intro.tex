\ssr{ВВЕДЕНИЕ}

В небесной механике – известна задача n тел, Которая заключается в описании относительного движения n материальных объектов, связанных друг с другом законом всемирного тяготения Ньютона. До сих пор общее решение задачи для трёх тел и более не получено.
Проблема не имеет решения в виде однозначных аналитических функций в общем случае, как, например, для двух тел \cite{solution3}. Объекты в задаче n тел представляются материальными точками, то есть точками в трёхмерном пространстве не имеющими размера, но имеющие массу, а их расстояние их взаимодействия существенно превышает их собственные размеры. Визуализация трёхмерного движения позволяет лучше представлять одно из фундаментальных взаимодействие в физике.

Целью данной работы является разработка программного обеспечения для визуализации задачи n тел в 3-х мерном пространстве в виде динамически-генерируемой анимации.

Для достижения этой цели необходимо выполнить следующие задачи:

\begin{enumerate}
	\item анализ физической задачи n тел и описание метода её решения;
	\item анализ существующих методов и алгоритмов для создания динамического изображения трёхмерной сцены;
	\item выбор наиболее подходящих алгоритмов для построения трёхмерной сцены;
	\item разработка выбранных алгоритмов;
	\item выбор средств реализации ПО;
	\item реализация ПО и выбранных алгоритмов;
	\item исследование погрешности выбранного численного метода решения задачи.
\end{enumerate}

\clearpage
