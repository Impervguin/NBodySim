\ssr{ВВЕДЕНИЕ}

В современном мире компьютерная графика является неотъемлемой частью многих сфер человеческой деятельностью. Она используется в художественных целях для визуализации различных сцен в кинематографе, компьютерных играх, при проектировании различных объектов в таких сферах как архитектура, ландшафтное проектрирование, дизайн интерьеров. В связи с этим перед специалистами встаёт задача создания реалистических 3-х мерных изображений, которые будут учитывать такие сложные детали как, оптичиские эффекта света, например, тени, преломление, рассеивание, должны учитывать цвет объектов, их текстуру. Но при этом алгоритмы компьютерной графики ресурсозатратны, поэтому часто приходится пренебрегать отдельными факторами реалистичности изображения для ускорения синтеза изображения.

Целью данной работы является разработка программного обеспечения для визуализации задачи n тел в 3-х мерном пространстве.

Для достижения этой цели необходимо выполнить следующие задачи:

\begin{enumerate}
	\item Рассмотрение физической задачи n тел и выбор метода её решения;
	\item Анализ существующих методов и алгоритмов для создания реалистического изображения трёхмерной сцены;
	\item Выбор наиболее подходящих алгоритмов для построения трёхмерной сцены;
	\item Проектирование архитектуры и графического интерфейса ПО;
	\item Выбор средств реализации ПО;
	\item Реализация ПО и выбранных алгоритмов;
	\item Проведение замеров временных характеристик алгоритмов построения сцены и решения задачи n тел.
\end{enumerate}

\clearpage
