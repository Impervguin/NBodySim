\ssr{ВВЕДЕНИЕ}

в небесной механике – известна задача n тел, Которая заключается в описании относительного движения n материальных объектов, связанных друг с другом законом всемирного тяготения Ньютона. До сих пор общее решение задачи трёх тел не получено.
Проблема не имеет решения в виде однозначных аналитических функций в общем случае, как, например, для двух тел \cite{solution3}. При этом в различных конфигурациях тела могут образовывать интересные, красивые зацикленные или не зацикленные траектории.

Целью данной работы является разработка программного обеспечения для визуализации задачи n тел в 3-х мерном пространстве в виде анимации.

Для достижения этой цели необходимо выполнить следующие задачи:

\begin{enumerate}
	\item Анализ физической задачи n тел и описание метода её решения;
	\item Анализ существующих методов и алгоритмов для создания динамического изображения трёхмерной сцены;
	\item Выбор наиболее подходящих алгоритмов для построения трёхмерной сцены;
	\item Проектирование архитектуры ПО;
	\item Выбор средств реализации ПО;
	\item Реализация ПО и выбранных алгоритмов;
	\item Проведение сравнения точного решения задачи для 2-х тел и решения, полученного программой.
\end{enumerate}

\clearpage
