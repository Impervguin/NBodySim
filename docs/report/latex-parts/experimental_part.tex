\chapter{Исследовательская часть}

В данной части будет проведён анализ погрешности выбранного численного метода решения задачи n тел.

\section{Физическая основа исследования}

По условию задачи, система состоит из n тел, при этом эти тела взаимодействуют только друг с другом и гравитационно, а сила всемирного тяготения является консервативной, значит система замкнута и имеет место закон сохранения энергии в форме~\ref{eq:energyeq}~\cite{fn4}.

\begin{equation}
	\label{eq:energyeq}
	E = E_k + E_p = const,
\end{equation}
где $E_k$ -- кинетическая энергия системы, а $E_p$ -- потенциальная.

Кинетическая энергия системы определяется как сумма кинетических энергий всех входящих её тел~\ref{eq:kinetic}.
\begin{equation}
	\label{eq:kinetic}
	E_k = \sum_{i=0}^n{\frac{m_iv_i^2}{2}},
\end{equation}
где $m_i$ -- масса i-го тела, а $v_i$ --величина его скорости.

Потенциальная энергия системы определяется как сумма потенциальных энергий всех взаимодействий, имеющих место в системе. В случае гравитационного взаимодействия энергия системы равна сумме энергий всех попарных взаимодействий тел ~\ref{eq:potential}~\cite{fn4}.


\begin{equation}
	\label{eq:potential}
	E_p = \sum_{i,j=0; i < j}^n{-G\frac{m_im_j}{R_{ij}}},
\end{equation}
где $G = 6.67430 * 10^{-11}$ -- гравитационная постоянная, константа, $R_{ij}$ -- расстояние между i-м и j-м телом.

Таким образом энергия системы в задаче должна сохраняться и отклонение от начальной энергии на каждом шаге численного метода может служить как метрика погрешности метода вычислений.

\section{Конфигурации}
В качестве тестовых конфигураций были выбраны три набора тестовых данных.

\textbf{Конфигурация 1}

Система состоит из 2-х тел находящихся в несовпадающих точках и не имеющих начальную скорость. Таким образом тела сближаются друг с другом до их столкновения.

\textbf{Конфигурация 2}

Система состоит из 4-х тел, каждый из которых лежит в плоскости $XoY$ и все они на разных направлениях данных осей, при этом их скорости направлены перпендикулярно осям, на которых они находятся. Параметры тел подобраны таким образом, что они движутся по спирали к центру, сближаясь до определённого момента, а затем по спирали отдаляются до позиций близких к исходным, и цикл повторяется.

\textbf{Конфигурация 3}

Система состоит из 1 тела с большой массой и 7 тел с меньшими, при этом начальные скорости заданы так, что они удовлетворяют условию движения по окружности~\ref{eq:normalacc} и все мелкие тела вращаются вокруг массивного в разных плоскостях. 
\begin{equation}
	\label{eq:normalacc}
	|\vec{a_n}| = \frac{|\vec{v}|^2}{R} \implies |\vec{v}| = \sqrt{\frac{Gm}{R}}
\end{equation}

\section{Результаты}

В таблице \ref{tbl:experiment} представлены результаты исследования погрешностей для каждой из конфигураций. Из результатов следует, что как при столкновении тел происходит большое увеличение погрешности вычислений, при этом для всех исследованных размерностей шага по времени.

Конфигурации 2 и 3 показывают, что при без столкновений погрешность метода Эйлера невелика даже при длительном просчёте.

\begin{longtable}{|r|r|r|r|r|r|}
	\caption{Результаты оценки погрешности} \label{tbl:experiment} 
	\\
	\hline
	\multirow{2}{0.08\textwidth}{\textbf{Время}} & \multirow{2}{0.19\textwidth}{\textbf{Шаг по времени}} &  \multirow{2}{0.17\textwidth}{\textbf{Конфигурация}} & \multicolumn{3}{|c|}{\textbf{Отклонение энергии}} \\
	\cline{4-6} & & &
	 \multicolumn{1}{|p{0.14\textwidth}|}{\textbf{Min, \%}} & \multicolumn{1}{|p{0.14\textwidth}|}{\textbf{Max, \%}} & \multicolumn{1}{|p{0.14\textwidth}|}{\textbf{Avg, \%}} \\
	\hline
	\endfirsthead
\hline
11 & 0.001 & 1 & 6.6743e-07 & 1.01249e+07 & 1.27481e+06 \\ \hline
11 & 0.0001 & 1 & 6.6743e-09 & 1.10058e+10 & 1.38693e+09 \\ \hline
11 & 0.00001 & 1 & 6.67456e-11 & 1.88086e+07 & 2.3705e+06 \\ \hline
11 & 0.000001 & 1 & 6.62998e-13 & 7.7732e+09 & 9.79688e+08 \\ \hline
11 & 0.0000001 & 1 & 0 & 3.74297e+08 & 4.71743e+07 \\ \hline
300 & 0.01 & 2 & 1.66799e-06 & 26.642 & 10.1957 \\ \hline
300 & 0.001 & 2 & 1.66799e-08 & 2.79235 & 1.21952 \\ \hline
300 & 0.0001 & 2 & 1.66797e-10 & 0.280533 & 0.123837 \\ \hline
10000 & 0.01 & 3 & 3.07938e-11 & 7.77897e-05 & 2.30798e-05 \\ \hline
10000 & 0.001 & 3 & 3.04754e-11 & 0.000583519 & 0.000287195 \\ \hline
10000 & 0.0001 & 3 & 2.03549e-12 & 0.00182867 & 0.00150004 \\ \hline

\end{longtable}

\section{Вывод}

В результате исследовательской части было выяснено, что использование явного метода эйлера даёт небольшую погрешность в вычислениях, кроме случаев столкновения тел.


\clearpage
