\chapter{Исследовательская часть}

В данной части будет проведён анализ погрешности выбранного численного метода решения задачи n тел.

\section{Физическая основа исследования}

По условию задачи, система состоит из n тел, при этом эти тела взаимодействуют только друг с другом и гравитационно, а сила всемирного тяготения является консервативной, значит система замкнута и имеет место закон сохранения энергии в форме~\ref{eq:energyeq}~\cite{fn4}.

\begin{equation}
	\label{eq:energyeq}
	E = E_k + E_p = const,
\end{equation}
где $E_k$ -- кинетическая энергия системы, а $E_p$ -- потенциальная.

Кинетическая энергия системы определяется как сумма кинетических энергий всех входящих её тел~\ref{eq:kinetic}.
\begin{equation}
	\label{eq:kinetic}
	E_k = \sum_{i=0}^n{\frac{m_iv_i^2}{2}},
\end{equation}
где $m_i$ -- масса i-го тела, а $v_i$ --величина его скорости.

Потенциальная энергия системы определяется как сумма потенциальных энергий всех взаимодействий, имеющих место в системе. В случае гравитационного взаимодействия энергия системы равна сумме энергий всех попарных взаимодействий тел ~\ref{eq:potential}~\cite{fn4}.


\begin{equation}
	\label{eq:potential}
	E_p = -\sum_{i,j=0; i < j}^n{G\frac{m_im_j}{R_{ij}}},
\end{equation}
где $G = 6.67430 * 10^{-11}$ -- гравитационная постоянная, константа, $R_{ij}$ -- расстояние между i-м и j-м телом.

Таким образом энергия системы в задаче должна сохраняться и отклонение от начальной энергии на каждом шаге численного метода может служить как метрика погрешности метода вычислений.

\section{Конфигурации}
В качестве тестовых конфигураций были выбраны три набора тестовых данных.

\textbf{Конфигурация 1}

Система состоит из 2-х тел находящихся в несовпадающих точках и не имеющих начальную скорость. Таким образом тела сближаются друг с другом до их столкновения.

\textbf{Конфигурация 2}

Система состоит из 4-х тел, каждый из которых лежит в плоскости $XoY$ и все они на разных направлениях данных осей, при этом их скорости направлены перпендикулярно осям, на которых они находятся. Параметры тел подобраны таким образом, что они движутся по спирали к центру, сближаясь до определённого момента, а затем по спирали отдаляются до позиций близких к исходным, и цикл повторяется.

\textbf{Конфигурация 3}

Система состоит из 1 тела с большой массой и 7 тел с  существенно меньшими, при этом начальные скорости заданы так, что они удовлетворяют условию движения по окружности~\ref{eq:normalacc} и все маломассивные тела вращаются вокруг массивного в разных плоскостях. 
\begin{equation}
	\label{eq:normalacc}
	|\vec{a_n}| = \frac{|\vec{v}|^2}{R} \implies |\vec{v}| = \sqrt{\frac{Gm}{R}}
\end{equation}

\section{Результаты}

В таблице \ref{tbl:experiment} представлены результаты исследования погрешностей для каждой из конфигураций.

Из результатов следует, что столкновении тел (конфигурация 1) вызывает большое увеличение погрешности, при этом это характерно для всех исследованных шагов по времени. Такой эффект был ожидаем, так как при сверхблизких дистанциях между объектами делается существенный шаг по времени, при этом на этом шагу учитывается большая сила, соответствующая сверхблизкому расстоянию.

При циклическом сближении и отдалении (конфигурация 2), метод эйлера показывает небольшую погрешность, которая при этом уменьшается вместе с уменьшением шага интегрирования.

При вращении множества маломассивных тел вокруг массивного (конфигурация) погрешность также несущественна, а также никакое из маломассивных тел не слетело с начальной круговой траектории. Однако, в данной конфигурации наблюдается увеличение погрешности при уменьшении шага по времени.


\begin{longtable}{|r|r|r|r|r|r|}
	\caption{Результаты оценки погрешности} \label{tbl:experiment} 
	\\
	\hline
	\multirow{2}{0.14\textwidth}{\textbf{Время, $0<t\neq T_{max}$}} & \multirow{2}{0.13\textwidth}{\textbf{Шаг по времени, $\tau$}} &  \multirow{2}{0.17\textwidth}{\textbf{Конфигурация}} & \multicolumn{3}{|c|}{\textbf{Отклонение энергии}} \\
	\cline{4-6} & & &
	 \multicolumn{1}{|p{0.14\textwidth}|}{\textbf{Min, \%}} & \multicolumn{1}{|p{0.14\textwidth}|}{\textbf{Max, \%}} & \multicolumn{1}{|p{0.14\textwidth}|}{\textbf{Avg, \%}} \\
	\hline
	\endfirsthead
	\caption{Результаты оценки погрешности}
	\\
	\hline
	\multirow{2}{0.14\textwidth}{\textbf{Время, $0<t\neq T_{max}$}} & \multirow{2}{0.13\textwidth}{\textbf{Шаг по времени, $\tau$}} &  \multirow{2}{0.17\textwidth}{\textbf{Конфигурация}} & \multicolumn{3}{|c|}{\textbf{Отклонение энергии}} \\
	\cline{4-6} & & &
	\multicolumn{1}{|p{0.14\textwidth}|}{\textbf{Min, \%}} & \multicolumn{1}{|p{0.14\textwidth}|}{\textbf{Max, \%}} & \multicolumn{1}{|p{0.14\textwidth}|}{\textbf{Avg, \%}} \\
	\hline
	\endhead
\hline
11 & 0.001 & 1 & 6.6743e-07 & 1.01249e+07 & 1.27481e+06 \\ \hline
11 & 0.0001 & 1 & 6.6743e-09 & 1.10058e+10 & 1.38693e+09 \\ \hline
11 & 0.00001 & 1 & 6.67456e-11 & 1.88086e+07 & 2.3705e+06 \\ \hline
11 & 0.000001 & 1 & 6.62998e-13 & 7.7732e+09 & 9.79688e+08 \\ \hline
11 & 0.0000001 & 1 & 0 & 3.74297e+08 & 4.71743e+07 \\ \hline
300 & 0.01 & 2 & 1.66799e-06 & 26.642 & 10.1957 \\ \hline
300 & 0.001 & 2 & 1.66799e-08 & 2.79235 & 1.21952 \\ \hline
300 & 0.0001 & 2 & 1.66797e-10 & 0.280533 & 0.123837 \\ \hline
10000 & 0.01 & 3 & 3.07938e-11 & 7.77897e-05 & 2.30798e-05 \\ \hline
10000 & 0.001 & 3 & 3.04754e-11 & 0.000583519 & 0.000287195 \\ \hline
10000 & 0.0001 & 3 & 2.03549e-12 & 0.00182867 & 0.00150004 \\ \hline

\end{longtable}

\section{Вывод}

В результате исследовательской части было проведено исследование погрешности механической энергии системы при решении задачи n тел численным методом интегрирования -- методом эйлера. В результате исследования погрешность в случаях без столкновения (сближения на сверхблизкие расстояния) тел при шаге интегрирования \textbf{0.0001} погрешность механической энергии в среднем составила от \textbf{0.0015\%} до \textbf{0.1238\%}, что для задачи визуализации является допустимыми величинами. при шаге интегрирования \textbf{0.001} погрешность механической энергии в среднем составила от \textbf{0.00028\%} до \textbf{1.21\%}, что также является допустимыми.

В случае столкновения тел наблюдался резкий скачок погрешности(\~10e8\%) независимо от шага интегрирования. Такой результат ожидаем для численного метода и является приемлемым для визуализации.

Таким образом в результате исследовательской части было выяснено, что разработанная модель имеет допустимую погрешность.


\clearpage
