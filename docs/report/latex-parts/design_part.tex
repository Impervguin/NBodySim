\chapter{Конструкторская часть}

В данной части будут разработаны функциональные схемы разрабатываемого ПО, а также разработаны структуры и алгоритмы, которые будут использованы в ПО.

\section{Функциональные схемы} 

На рисунках~\ref{fig:A0}~--~\ref{fig:A3} представлено формальное описании разрабатываемого ПО в виде idef0-диаграммы.

\begin{figure}[H]
	\centering
	\includesvg[width=0.9\textwidth]{images/ramus/01_A0}
	\caption{<<Контекстная диаграмма верхнего уровня в нотации idef0>>}
	\label{fig:A0}
\end{figure}

\begin{figure}[H]
	\centering
	\includesvg[width=0.9\textwidth]{images/ramus/02_A0}
	\caption{<<Основной цикл ПО в нотации idef0>>}
	\label{fig:A1}
\end{figure}

\begin{figure}[H]
	\centering
	\includesvg[width=0.9\textwidth]{images/ramus/03_A3}
	\caption{<<Построение кадра в нотации idef0>>}
	\label{fig:A3}
\end{figure}

\section{Модель камеры}

Камера задаётся своим положением, направлением взгляда и направлением вверх и пирамидой видимости. Вектор в сторону получается векторным произведением вектора взгляда на вектор вверх. В пространстве камеры направление взгляда совпадает с положительным направлением оси z, а направление вверх -- с положительным направлением оси y.

Пирамида видимости -- усечённая 4-х гранная пирамида, уходящая в бесконечность, основание которой параллельно XoY. Основание пирамиды задаётся 3-мя параметрами:

\begin{itemize}
	\item $d$ -- расстояние от положения камеры до основания;
	\item $rx$ -- полуширина основания;
	\item $ry$ -- полувысота основания.
\end{itemize}

Пирамида видимости в пространстве камеры продемонстрирована на рисунке~\ref{fig:camera}.

\begin{figure}[H]
	\centering
	\includegraphics[width=0.5\textwidth]{camera}
	\caption{<<Пирамида видимости камеры>>}
	\label{fig:camera}
\end{figure}

Перспективная проекция по пирамиде видимости рассчитывается по формуле~\ref{eq:perspective}~\cite{projection}.

\begin{equation}
	\label{eq:perspective}
	\begin{pmatrix}
		x_p \\
		y_p \\
		z_p \\
		w_p \\
	\end{pmatrix} = 
	\begin{pmatrix}
		\frac{d}{rx}  & 0 & 0 & 0 \\
		0 & \frac{d}{ry} & 0 & 0 \\
		0 & 0 & 1 & 2d \\
		0 & 0 & 1 & 0 \\
	\end{pmatrix}
	\begin{pmatrix}
		x \\
		y \\
		z \\
		w \\
	\end{pmatrix}.
\end{equation}

Преобразование к пространству камеры выполняется по формуле~\ref{eq:view}~\cite{view}.

\begin{equation}
	\label{eq:view}
	\begin{pmatrix}
		x_v \\
		y_v \\
		z_v \\
		w_v \\
	\end{pmatrix} = 
	\begin{pmatrix}
		lx & ux & fx & 0 \\
		ly & uy & fy & 0 \\
		lz & uz & fz & 0 \\
		0 & 0 & 0 & 1 \\
	\end{pmatrix}
	\begin{pmatrix}
		x \\
		y \\
		z \\
		w \\
	\end{pmatrix},
\end{equation}

где
\begin{itemize}
	\item $\vec{u} = (ux, uy, uz)^T$ -- вектор взгляда;
	\item $\vec{f} = (fx, fy, fz)^T$ -- вектор вверх;
	\item $\vec{l} = (lx, ly, lz)^T = [\vec{u}, \vec{f}]$ -- вектор влево;
\end{itemize}

\section{Разработка типов и структур данных}

В работе и для разработки алгоритмов необходимы следующие типы и структуры данных.

\begin{enumerate}
	\item Математические абстракции
	\begin{enumerate}
		\item вектор -- задаётся 3-мя вещественными числами -- координатами вектора;
		\item плоскость:
		\begin{itemize}
			\item $normal$ -- вектор внутренней (если задано направление) нормали плоскости;
			\item $d$ -- свободный член в уравнение плоскости;
		\end{itemize}
	\end{enumerate}
	\item Физическое тело:
	\begin{itemize}
		\item $pos$ -- положение тела, заданное вектором;
		\item $vel$ -- вектор скорости тела;
		\item $mass$ -- масса;
	\end{itemize}
	\item Полигон:
	\begin{itemize}
		\item $v1$, $v2$, $v3$ -- вектора, задающие точки образующие полигон;
		\item $normal$ -- внутренняя (если задано направление) нормаль плоскости, проходящей через полигон;
	\end{itemize}
	\item Полигональный объект:
	\begin{itemize}
		\item $vertices$ -- массив векторов, задающих точки образующие объект;
		\item $polygons$ -- массив полигонов, из которых состоит объект;
		\item $pos$ -- вектор, задающий положение объекта;
	\end{itemize}
	\item Точечный источник света:
	\begin{itemize}
		\item $pos$ -- вектор, задающий положение источника света;
		\item $intensity$ -- интенсивность или цвет света, задаваемый 3-мя беззнаковыми числами: R, G и B, задающие соответственно красный, зелёный и синий цвета;
	\end{itemize}
	\item Камера:
	\begin{itemize}
		\item $pos$ -- вектор, задающий положение камеры;
		\item $forward$ -- вектор, задающий направление взгляда;
		\item $up$ -- вектор, задающий направление вверх;
		\item $px$ -- полуширина окна камеры;
		\item $py$ -- полувысота окна камеры;
		\item $d$ -- расстояние от камеры, до окна камеры.
	\end{itemize}
\end{enumerate}


\section{Разработка алгоритма решения задачи n тел}

\textbf{Алгоритм расчёта положения тел}

\textbf{Входные данные:}
\begin{itemize}
	\item массив тел $arr$;
	\item промежуток времени $dt$, заданный вещественным числом.
\end{itemize}

\textbf{Выходные данные:}
\begin{itemize}
	\item массив обновлённых тел $out\_arr$.
\end{itemize}

\textbf{Алгоритм}
\begin{enumerate}
	\item создать пустой массив тел $out\_arr$ размера массива $arr$;
	\item для каждого тела $body$ из массива $arr$:
	\begin{enumerate}
		\item $dp = $ произведение вектора $vel$ на скаляр $dt$;
		\item $npos = pos + dp$
		\item $force = $ нулевой вектор;
		\item для каждого тела $infl$ из массива $arr$:
		\begin{enumerate}
			\item если $infl$ не равен $body$:
			\item прибавить к $force$ силу действующую на  $body$ телом $infl$;
		\end{enumerate}
		\item $nvel = vel + $ скалярное произведение $force$ на $\frac{dt}{mass}$;
		\item $nbody$ -- новое тело с параметрами ($npos$, $nvel$, $mass$);
		\item добавить $nbody$ в $out_arr$;
	\end{enumerate}
	\item возврат $out\_arr$.
\end{enumerate}

\textbf{Алгоритм расчёта силы, действующей одним телом на другое}
\textbf{Входные данные:}
\begin{itemize}
	\item $body$ -- тело, на которое действует сила;
	\item $infl$ -- тело, которое действует на $body$;
	\item каждое тело задано:
	\begin{itemize}
		\item $pos$ -- точка положения тела;
		\item $vel$ -- вектор скорости тела;
		\item $mass$ -- масса тела;
	\end{itemize}
\end{itemize}

\textbf{Выходные данные:}
\begin{itemize}
	\item $force$ -- вектор силы, действующий на $body$ со стороны $infl$.
\end{itemize}

\textbf{Алгоритм}

\begin{enumerate}
	\item вектор $diff$ -- разность векторов $infl.pos$ и $body.pos$;
	\item $distance$ -- квадрат расстояние между $body$ и $infl$ -- $diff^2$;
	\item $force$ -- нормализованный вектор $diff$;
	\item умножить $force$ на скаляр $G=6.67430e^{-11} * infl.mass * body.mass / distance$;
	\item возврат $force$.
\end{enumerate}

\section{Разработка алгоритма отсечения частей вне видимости камеры}

\textbf{Алгоритм отсечения полигона по плоскости}

\textbf{Входные данные:}
\begin{itemize}
	\item $plane$ -- плоскость;
	\item $poly$ -- полигон.
\end{itemize}

\textbf{Выходные данные:}
\begin{itemize}
	\item $polys$ -- массив отсечённых полигонов;
	\item $seen$ -- флаг, что полигон виден целиком или частично относительно плоскости;
\end{itemize}

\textbf{Алгоритм}

\begin{enumerate}
	\item создать пустой массив полигонов $polys$ размером массива 2;
	\item создать пустые массивы точек $see$ и $nsee$ размерами 3;
	\item если скалярное произведение $plane.normal$ и $poly.v1$ + $plane.d$ >= 0:
	\begin{enumerate}
		\item добавить $v1$ в массив $see$;
	\end{enumerate}
	\item иначе:
	\begin{enumerate}
		\item добавить $v1$ в массив $nsee$;
	\end{enumerate}
	\item аналогично для $v2$ и $v3$;
	
	\item если размер $see$ = 0:
	\begin{enumerate}
		\item вернуть $polys$, $false$;
	\end{enumerate}
	\item иначе если размер $see$ = 3:
	\begin{enumerate}
		\item добавить $poly$ в $polys$
	\end{enumerate}
	\item иначе если размер $see$ = 2:
	\begin{enumerate}
		\item $v1$, $v2$, $v3$ = $see[0]$, $see[1]$, $nsee[0]$;
		\item добавить в $polys$ полигон, образованный Алгоритм заключается в последовательном отсечении каждого полигона объекта по каждой из плоскостей.$v1$, $v2$ и точкой пересечения ребра $v1v3$ с плоскостью;
		\item добавить в $polys$ полигон, образованный $v2$ и точками пересечений рёбер $v1v3$ и $v2v3$ с плоскостью;
	\end{enumerate}
	\item иначе
	\begin{enumerate}
		\item добавить в $polys$ полигон, образованный видимой точкой и точками пересечения рёбер между видимой точки и невидимыми;
	\end{enumerate}
	\item вернуть $polys$, $true$
\end{enumerate}


\textbf{Алгоритм отсечения частей объектов вне поля видимости камеры}

При отсечении по зоне видимости используются плоскости, образующие пирамиду видимости. Каждая такая плоскость описывается парой: внутренняя нормаль и скалярное значение. Методом, описанном в \cite{gabriella}, были найдены плоскости для используемой модели камеры:
	
\begin{itemize}
	\item задняя -- $((0, 0, 1), -d)$;
	\item левая -- $((\frac{d}{px}, 0, 1), 0)$;
	\item правая -- $((-\frac{d}{px}, 0, 1), 0)$;
	\item нижняя -- $((0, \frac{d}{py}, 1), 0)$;
	\item верхняя -- $((0, -\frac{d}{py}, 1), 0)$;
\end{itemize}

Алгоритм отсечения частей объектов вне поля видимости камеры заключается в последовательном отсечении каждого полигона объекта по каждой из плоскостей, по описанному выше алгоритму.


\section{Разработка алгоритма z-буфера}

\textbf{Алгоритм Z-буфера}

\textbf{Входные данные:}
\begin{itemize}
	\item $objs$ -- массив полигональных объектов.
\end{itemize}

\textbf{Выходные данные:}
\begin{itemize}
	\item $depthbuf$ -- буфер глубины, матрица размерами экрана вещественных чисел;
	\item $screebbuf$ -- буфер экрана, матрица размерами экрана цветов;
	\item $normalbuf$ -- буфер нормалей, матрица размерами экрана векторов нормалей к поверхностям.
\end{itemize}

\begin{enumerate}
	\item Инициализировать значение буфера кадра фоновым цветом;
	\item  Инициализировать буфера глубины максимальным значением вещественных чисел;
	\item Для каждого объекта $obj$ из $objs$:
	\begin{enumerate}
		\item для каждого полигона $poly$ из $obj.polygons$:
		\begin{enumerate}
			\item растеризовать $poly$;
			\item для каждого пикселя $pixel$ из растеризованного $poly$, если глубина $pixel$ меньше, чем глубина соответствующего $pixel$ значения из буфера глубины:
			\item обновить значение цвета в соответствущей $pixel$ ячейки из $screenbuf$;
			\item обновить нормаль соответствущей $pixel$ ячейки из $screenbuf$ на нормаль $poly$;
		\end{enumerate}
	\end{enumerate}
	\item Вернуть $screenbuf$, $depthbuf$ и $normalbuf$.
\end{enumerate}

\section{Разработка алгоритма теневого буфера}

\textbf{Алгоритм создания теневого буфера}

\textbf{Входные данные:}
\begin{itemize}
	\item $light$ -- точечный источник света;
	\item $size$ -- размер теневых карт;
	\item $objs$ -- массив полигональных объектов.
\end{itemize}

\textbf{Выходные данные:}
\begin{itemize}
	\item $shadows$ -- массив теневых карт.
\end{itemize}

\textbf{Алгоритм}
\begin{enumerate}
	\item Для каждой из 6-ти сторон куба, описывающего источник света:
	\begin{enumerate}
		\item инициализировать теневую карту буфером глубины размером $size$ на $size$;
		\item преобразовать объекты $objs$ к пространству экрана стороны куба;
		\item построить карту теней алгоритмом z-буфера;
		\item добавить карту теней в массив $shadows$;
	\end{enumerate}
	\item Вернуть $shadows$.
\end{enumerate}


\textbf{Алгоритм проверки точки в теневом буфере}

\textbf{Входные данные:}
\begin{itemize}
	\item $point$ -- точка для проверки;
	\item $shadows$ -- массив теневых карт.
\end{itemize}

\textbf{Выходные данные:}
\begin{itemize}
	\item $in\_shadow$ -- флаг, находится ли $point$ в тени.
\end{itemize}

\textbf{Алгоритм}
\begin{enumerate}
	\item $in\_shadows$ = $false$;
	\item Для каждой теневой карты из $shadows$:
	\begin{enumerate}
		\item преобразовать $point$ к пространству теневой карты;
		\item если глубина $point$ больше, чем глубина соответствующей ячейки из теневой карты:
		\item $in\_shadows$ = $true$;
	\end{enumerate}
	\item Вернуть $in\_shadows$.
\end{enumerate}


\section*{Вывод}

В результате конструкторской части было создано формальное описание разрабатываемого ПО, определены типы и структуры, необходимые для разработки ПО, а также разработаны основные алгоритмы, которые будут применены в разрабатываемом ПО.

\clearpage
