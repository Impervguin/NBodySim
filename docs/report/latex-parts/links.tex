\addcontentsline{toc}{chapter}{СПИСОК ИСПОЛЬЗОВАННЫХ ИСТОЧНИКОВ}
\begin{thebibliography}{}
	\bibitem{markeev} Маркеев, А. П. ЗАДАЧА ТРЕХ ТЕЛ И ЕЕ ТОЧНЫЕ РЕШЕНИЯ [Текст] / А. П. Маркеев // Соровский образовательный журнал. — 1999. — № 9. — С. 112-117.
	\bibitem{solution3} Рубинштейн А.И., Городецкая Т.А., Серебренников П.С., Шипов Н.В., Шмаков А.В. ОБ ОДНОМ ЧАСТНОМ СЛУЧАЕ РЕШЕНИЯ ЗАДАЧИ ТРЕХ ТЕЛ // Проблемы современной науки и образования. - Иваново: Проблемы науки, 2017. - С. 6-9.
	\bibitem{nbody-numeric} SOME THEORETICAL AND NUMERICAL ASPECTS OF THE N-BODY PROBLEM // LUND UNIVERSITY LIBRARIES. URL: https://lup.lub.lu.se/student-papers/search/publication/4780668 (дата обращения: 12.09.2024).
	\bibitem{samarskii} Самарский А. А. ВВЕДЕНИЕ В ЧИСЛЕННЫЕ МЕТОДЫ. - СПб: Лань, 2009. - 288 с.
	\bibitem{platon-body} Правильные многогранники. // Большая Российская энциклопедия URL: https://bigenc.ru/c/pravil-nye-mnogogranniki-6cab23?ysclid=m3t3ptp0y231309393 (дата обращения: 12.09.2024).
	\bibitem{kazancev} Казанцев А.В. Основы компьютерной графики. Часть 1. Математический аппарат компьютерной графики. - Казань: 2001. - 62 с.
	\bibitem{gabriella} Гэбриел Гамбетта. Компьютерная графика. Рейтрейсинг и растеризация. — СПб.: Питер, 2022. — 224 с.: ил. — (Серия «Библиотека программиста»).
	\bibitem{ngtu} Лекция 9. Удаление невидимых линий. // НГТУ - компьютерные голографические измерительные системы URL: http://optic.cs.nstu.ru/files/CC/CompGraph/L6\_удаление невидимых линий.pdf (дата обращения: 30.11.2024).
	\bibitem{colins} On Vertex-Vertex Systems and Their Use in Geometric and Biological Modelling : Диссертация на соискание доктора технических наук / Colin Smith ; THE UNIVERSITY OF CALGARY. — Calgary, 2006. — 204 c.
	\bibitem{rodgers} Роджерс Д. Алгоритмические основы машинной графики [Текст] / Роджерс Д. — Москва: Мир, 1989 — 512 c.
	\bibitem{design-patterns} Design Patterns // Refactoring.Guru URL: https://refactoring.guru (дата обращения: 22.10.2024).
\end{thebibliography}
